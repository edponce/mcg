\section{Introduction}
\label{sec:introduction}

Krylov subspace iterative methods~\cite{iterativeGPU} are among the most popular methods for
solving large sparse linear systems of the form $Ax = b$. Specifically, the conjugate gradient
Krylov methods have been studied and used extensively due to their stability and accuracy.
The computing performance demands and interest in larger problems,
has led to the implementation of numerous software libraries containing variations of
conjugate gradients methods. A common performance issue is the parallelization of preconditioners.
A common communication issue is the calculation of global inner products since a pair of
global communications are required~\cite{gu2004msdcg}. The inner products serve as synchronization points
in parallel implementations. Alternatives are domain decomposition methods of
addivite Schwarz type~\cite{tang1992schwarz} but convergence degrades with increase subdomains.
Another alternative is to replace global inner products with a small linear system of equations that
is solved iteratively and communication required only between neighbouring subdomain~\cite{gu2004msdcg}.
This approximates MSD-CG method as global inner product free CG (GPIF-CG).


